\documentclass[a4paper, 14pt]{extarticle}
\usepackage[settings]{markdown}
\usepackage{minted}

% Поля
%--------------------------------------
\usepackage{geometry}
\geometry{a4paper,tmargin=2cm,bmargin=2cm,lmargin=3cm,rmargin=1cm}
%--------------------------------------


%Russian-specific packages
%--------------------------------------
\usepackage[T2A]{fontenc}
\usepackage[utf8]{inputenc} 
\usepackage[english, main=russian]{babel}
%--------------------------------------

\usepackage{textcomp}

% Красная строка
%--------------------------------------
\usepackage{indentfirst}               
%--------------------------------------             


%Graphics
%--------------------------------------
\usepackage{graphicx}
\graphicspath{ {./images/} }
\usepackage{wrapfig}
%--------------------------------------

% Полуторный интервал
%--------------------------------------
\linespread{1.3}                    
%--------------------------------------

%Выравнивание и переносы
%--------------------------------------
% Избавляемся от переполнений
\sloppy
% Запрещаем разрыв страницы после первой строки абзаца
\clubpenalty=10000
% Запрещаем разрыв страницы после последней строки абзаца
\widowpenalty=10000
%--------------------------------------

%Списки
\usepackage{enumitem}

%Подписи
\usepackage{caption} 

%Гиперссылки
\usepackage{hyperref}

\hypersetup {
	unicode=true
}

%Рисунки
%--------------------------------------
\DeclareCaptionLabelSeparator*{emdash}{~--- }
\captionsetup[figure]{labelsep=emdash,font=onehalfspacing,position=bottom}
%--------------------------------------

\usepackage{tempora}
\usepackage{amsmath}
\usepackage{color}
\usepackage{listings}
\lstset{
  belowcaptionskip=1\baselineskip,
  breaklines=true,
  frame=L,
  xleftmargin=\parindent,
  language=Python,
  showstringspaces=false,
  basicstyle=\footnotesize\ttfamily,
  keywordstyle=\bfseries\color{blue},
  commentstyle=\itshape\color{purple},
  identifierstyle=\color{black},
  stringstyle=\color{red},
}

%--------------------------------------
%			НАЧАЛО ДОКУМЕНТА
%--------------------------------------

\begin{document}

%--------------------------------------
%			ТИТУЛЬНЫЙ ЛИСТ
%--------------------------------------
\begin{titlepage}
\thispagestyle{empty}
\newpage


%Шапка титульного листа
%--------------------------------------
\vspace*{-60pt}
\hspace{-65pt}
\begin{minipage}{0.3\textwidth}
\hspace*{-20pt}\centering
\includegraphics[width=\textwidth]{emblem}
\end{minipage}
\begin{minipage}{0.67\textwidth}\small \textbf{
\vspace*{-0.7ex}
\hspace*{-6pt}\centerline{Министерство науки и высшего образования Российской Федерации}
\vspace*{-0.7ex}
\centerline{Федеральное государственное бюджетное образовательное учреждение }
\vspace*{-0.7ex}
\centerline{высшего образования}
\vspace*{-0.7ex}
\centerline{<<Московский государственный технический университет}
\vspace*{-0.7ex}
\centerline{имени Н.Э. Баумана}
\vspace*{-0.7ex}
\centerline{(национальный исследовательский университет)>>}
\vspace*{-0.7ex}
\centerline{(МГТУ им. Н.Э. Баумана)}}
\end{minipage}
%--------------------------------------

%Полосы
%--------------------------------------
\vspace{-25pt}
\hspace{-35pt}\rule{\textwidth}{2.3pt}

\vspace*{-20.3pt}
\hspace{-35pt}\rule{\textwidth}{0.4pt}
%--------------------------------------

\vspace{1.5ex}
\hspace{-35pt} \noindent \small ФАКУЛЬТЕТ\hspace{50pt} <<Информатика и системы управления>>

\vspace*{-16pt}
\hspace{47pt}\rule{0.83\textwidth}{0.4pt}

\vspace{0.5ex}
\hspace{-35pt} \noindent \small КАФЕДРА\hspace{50pt} <<Теоретическая информатика и компьютерные технологии>>

\vspace*{-16pt}
\hspace{30pt}\rule{0.866\textwidth}{0.4pt}
  
\vspace{11em}

\begin{center}
\Large {\bf Лабораторная работа № 1} \\ 
\large {\bf по курсу <<Распределение параллельных и распределённых программ>>}\\
\large <<Pаспараллеливание алгоритма вычисления произведения двух матриц>>
\end{center}\normalsize

\vspace{8em}


\begin{flushright}
  {Студент группы ИУ9-51Б Горбунов А. Д.\hspace*{15pt} \\
  \vspace{2ex}
  Преподаватель Царёв А. С.\hspace*{15pt}}
\end{flushright}

\bigskip

\vfill
 

\begin{center}
\textsl{Москва 2024}
\end{center}
\end{titlepage}
%--------------------------------------
%		КОНЕЦ ТИТУЛЬНОГО ЛИСТА
%--------------------------------------

\renewcommand{\ttdefault}{pcr}

\setlength{\tabcolsep}{3pt}
\newpage
\setcounter{page}{2}

\section{Задача}\label{Sect::task}
\par
    Замерить время вычисления, сравнить с временем при вычислении элементов матрицы С не по строкам, а по столбцам. Размер матриц подобрать таким образом, чтобы время выполнения на вашей машине было не слишком непоказательно малым (меньше нескольких минут), но и не чересчур большим (несколько часов). Использовать библиотечные функции для вычисления произведений матриц нельзя.

 
    Затем конечную матрицу С условно разделить на примерно равные прямоугольные подматрицы и распараллелить программу таким образом, чтобы каждый поток занимался вычислением своей подматрицы. Матрицы А и В для этого разделить на примерно равные группы строк и столбцов соответственно. Сделать для разного количества потоков (разных разбиений), также замерить время вычисления, сравнить с вычислениями стандартным алгоритмом. Также по окончании вычислений сравнивать получившуюся матрицу с той, что была вычислена стандартным алгоритмом, для проверки правильности вычислений.

\section{Характеристики устройства}\label{Sect::task}

\begin{figure}
    \centering
    \includegraphics[width=1\linewidth]{picture_st.png}
    \label{fig:enter-label}
\end{figure}

\pagebreak
\section{Код решения}
\begin{minted}{python}
    Файл main.py
import random
import numpy as np
import threading, time
n_1 = 501  # Размерность матрицы
m_1 = 520
matrix_1 = np.array(
[[random.randint(-1000, 1000) for _ in range(m_1)] for _ in range(n_1)])

n_2 = 520  # Размерность матрицы
m_2 = 515
matrix_2 = np.array(
[[random.randint(-1000, 1000) for _ in range(m_2)] for _ in range(n_2)])

print("ожидаемый результат:")
print(np.matmul(matrix_1, matrix_2))

def multiplicationOfCube(n, final_matrix):
    for i in range(n):
        for j in range(m_2):
            c = 0
            for k in range(len(matrix_1[i])):
                c += matrix_1[i][k] * matrix_2[j][k]
            final_matrix[i, j] = c
    
def multiplicationOfCubeThreading(n, h, b, final_matrix):
    n_min = 0
    if h > 0:
        if h == 1:
            n_min = n-1
        else:
            n_min = b*(h-1)*n
        n = b*h*n
    if n == 0:
        n = n_1

    for i in range(n_min, n):
        if i >= len(matrix_1):
            break
        for j in range(m_2):
            c = 0
            for k in range(len(matrix_2[j])):
                c += matrix_1[i][k] * matrix_2[j][k]
            final_matrix[i, j] = c


def NThreadings(b):
    final_matrix = np.array(
    [[random.randint(-1000, 1000) for _ in range(m_2)] for _ in range(n_1)])
    #print(final_matrix)

    for h in range(b):
        if n_1 %2 !=0:
            thread = threading.Thread(name='worker', 
                                target=multiplicationOfCubeThreading, 
                                args=(n_1//b +1, h, b, final_matrix,))
        else:
            thread = threading.Thread(name='worker', 
                                target=multiplicationOfCubeThreading, 
                                args=(n_1//b, h, b, final_matrix,))
        thread.start()
    
    t_start = time.time()
    while any(th.is_alive() 
              for th in threading.enumerate() 
              if th.name == 'worker'):
        continue
    
    tm = round(time.time() - t_start, 2)
    print(f'{b} thread время: {tm}')
    print(final_matrix)


if __name__ == '__main__':
    print("1-я матрица:")
    print(matrix_1)
    print("2-я матрица:")
    print(matrix_2)
    matrix_2 = matrix_2.T

    final_matrix = np.array(
    [[random.randint(-1000, 1000) for _ in range(m_2)] for _ in range(n_1)])

    print("заготовка финальной матрицы:")
    print(final_matrix)

    t_start = time.time()

    multiplicationOfCube(n_1, final_matrix)

    tm = round(time.time() - t_start, 2)

    print(f'1 thread время: {tm}')
    print(final_matrix)
    
    NThreadings(1)
    NThreadings(2)
    NThreadings(4)
    NThreadings(6)
    NThreadings(8)
    NThreadings(10)
\end{minted}

\section{Времена работы программы}

\begin{tabular}
{|p{10cm}|p{5cm}|}
\hline
Потоки & Время \\
\hline
1(без использования потоков) & 76.43 с \\
\hline
1(с использованием потоков) & 380.94 с \\
\hline
2 & 295.35 с \\
\hline
4 & 251.84 с \\
\hline
6 & 285.46 с \\
\hline
8 & 341.22 с \\
\hline
10 & 370.03 с \\
\hline
\end{tabular}


\section{Заключение}

    В данной работе я изучил возможности языка python в работе с библиотекой threading, а именно научился распределять вычисления на потоки.

\section{Результат запуска}
    
\begin{figure}[!htb]
	\centering
	\includegraphics[width=0.8\textwidth]{picture_1.png}
\label{fig:picture_1.png}
\end{figure}

\begin{figure}[!htb]
	\centering
	\includegraphics[width=0.8\textwidth]{picture_2.png}
\label{fig:picture_2.png}
\end{figure}

\end{document}

