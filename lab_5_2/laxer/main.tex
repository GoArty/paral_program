\documentclass[a4paper, 14pt]{extarticle}
\usepackage[settings]{markdown}
\usepackage{minted}

% Поля
%--------------------------------------
\usepackage{geometry}
\geometry{a4paper,tmargin=2cm,bmargin=2cm,lmargin=3cm,rmargin=1cm}
%--------------------------------------


%Russian-specific packages
%--------------------------------------
\usepackage[T2A]{fontenc}
\usepackage[utf8]{inputenc} 
\usepackage[english, main=russian]{babel}
%--------------------------------------

\usepackage{textcomp}

% Красная строка
%--------------------------------------
\usepackage{indentfirst}               
%--------------------------------------             


%Graphics
%--------------------------------------
\usepackage{graphicx}
\graphicspath{ {./images/} }
\usepackage{wrapfig}
%--------------------------------------

% Полуторный интервал
%--------------------------------------
\linespread{1.3}                    
%--------------------------------------

%Выравнивание и переносы
%--------------------------------------
% Избавляемся от переполнений
\sloppy
% Запрещаем разрыв страницы после первой строки абзаца
\clubpenalty=10000
% Запрещаем разрыв страницы после последней строки абзаца
\widowpenalty=10000
%--------------------------------------

%Списки
\usepackage{enumitem}

%Подписи
\usepackage{caption} 

%Гиперссылки
\usepackage{hyperref}

\hypersetup {
	unicode=true
}

%Рисунки
%--------------------------------------
\DeclareCaptionLabelSeparator*{emdash}{~--- }
\captionsetup[figure]{labelsep=emdash,font=onehalfspacing,position=bottom}
%--------------------------------------

\usepackage{tempora}
\usepackage{amsmath}
\usepackage{color}
\usepackage{listings}
\lstset{
  belowcaptionskip=1\baselineskip,
  breaklines=true,
  frame=L,
  xleftmargin=\parindent,
  language=Python,
  showstringspaces=false,
  basicstyle=\footnotesize\ttfamily,
  keywordstyle=\bfseries\color{blue},
  commentstyle=\itshape\color{purple},
  identifierstyle=\color{black},
  stringstyle=\color{red},
}

%--------------------------------------
%			НАЧАЛО ДОКУМЕНТА
%--------------------------------------

\begin{document}

%--------------------------------------
%			ТИТУЛЬНЫЙ ЛИСТ
%--------------------------------------
\begin{titlepage}
\thispagestyle{empty}
\newpage


%Шапка титульного листа
%--------------------------------------
\vspace*{-60pt}
\hspace{-65pt}
\begin{minipage}{0.3\textwidth}
\hspace*{-20pt}\centering
\includegraphics[width=\textwidth]{emblem}
\end{minipage}
\begin{minipage}{0.67\textwidth}\small \textbf{
\vspace*{-0.7ex}
\hspace*{-6pt}\centerline{Министерство науки и высшего образования Российской Федерации}
\vspace*{-0.7ex}
\centerline{Федеральное государственное бюджетное образовательное учреждение }
\vspace*{-0.7ex}
\centerline{высшего образования}
\vspace*{-0.7ex}
\centerline{<<Московский государственный технический университет}
\vspace*{-0.7ex}
\centerline{имени Н.Э. Баумана}
\vspace*{-0.7ex}
\centerline{(национальный исследовательский университет)>>}
\vspace*{-0.7ex}
\centerline{(МГТУ им. Н.Э. Баумана)}}
\end{minipage}
%--------------------------------------

%Полосы
%--------------------------------------
\vspace{-25pt}
\hspace{-35pt}\rule{\textwidth}{2.3pt}

\vspace*{-20.3pt}
\hspace{-35pt}\rule{\textwidth}{0.4pt}
%--------------------------------------

\vspace{1.5ex}
\hspace{-35pt} \noindent \small ФАКУЛЬТЕТ\hspace{50pt} <<Информатика и системы управления>>

\vspace*{-16pt}
\hspace{47pt}\rule{0.83\textwidth}{0.4pt}

\vspace{0.5ex}
\hspace{-35pt} \noindent \small КАФЕДРА\hspace{50pt} <<Теоретическая информатика и компьютерные технологии>>

\vspace*{-16pt}
\hspace{30pt}\rule{0.866\textwidth}{0.4pt}
  
\vspace{11em}

\begin{center}
\Large {\bf Лабораторная работа № 5\_2} \\ 
\large {\bf по курсу <<Распределение параллельных и распределённых программ>>}\\
\large <<Синхронизация потоков>>
\end{center}\normalsize

\vspace{8em}


\begin{flushright}
  {Студент группы ИУ9-51Б Горбунов А. Д.\hspace*{15pt} \\
  \vspace{2ex}
  Преподаватель Царёв А. С.\hspace*{15pt}}
\end{flushright}

\bigskip

\vfill
 

\begin{center}
\textsl{Москва 2024}
\end{center}
\end{titlepage}
%--------------------------------------
%		КОНЕЦ ТИТУЛЬНОГО ЛИСТА
%--------------------------------------

\renewcommand{\ttdefault}{pcr}

\setlength{\tabcolsep}{3pt}
\newpage
\setcounter{page}{2}

\section{Код программы}
\begin{minted}{c++}
                                        Файл main.cpp:
#include <iostream>
#include <thread>
#include <vector>
#include <shared_mutex>
#include <mutex>
#include <unordered_set>
#include <random>
#include <list>

using namespace std;

shared_mutex list_mutex;
list<int> shared_list;
unordered_set<int> unique_numbers;

void generate_and_insert(int num_threads, int num_numbers) {
    random_device rd;
    mt19937 gen(rd());
    uniform_int_distribution<> dis(0, 1000);

    for (int i = 0; i < num_numbers; i++) {
        int number = dis(gen);

        shared_lock<std::shared_mutex> read_lock(list_mutex);
        if (unique_numbers.find(number) == unique_numbers.end()) {
            read_lock.unlock();

            unique_lock<std::shared_mutex> write_lock(list_mutex);
            if (unique_numbers.find(number) == unique_numbers.end()) {
                shared_list.push_back(number);
                unique_numbers.insert(number);
            }
        }
        
    }
}

bool check_for_duplicates() {
    shared_lock<shared_mutex> read_lock(list_mutex);
    unordered_set<int> temp_set;
    for (const auto& number : shared_list) {
        if (temp_set.find(number) != temp_set.end()) 
            return false;
        temp_set.insert(number);
    }
    return true;
}

void print_list() {
    shared_lock<shared_mutex> read_lock(list_mutex);
    cout << "List contents: ";
    for (const auto& number : shared_list) 
        cout << number << " ";
    cout << std::endl;
}

int main() {
    int num_threads = 4;
    int num_numbers_per_thread = 10000;

    vector<thread> threads;
    for (int i = 0; i < num_threads; i++) 
        threads.emplace_back(generate_and_insert, num_threads, num_numbers_per_thread);

    for (auto& thread : threads) 
        thread.join();
    
    print_list();

    if (check_for_duplicates()) 
        cout << "No duplicates found in the list." << endl;
    else 
        cout << "Duplicates found in the list." << endl;

    return 0;
}
\end{minted}

\end{document}

